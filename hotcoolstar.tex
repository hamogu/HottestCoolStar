%% Beginning of file 'sample631.tex'
%%
%% Modified 2021 March
%%
%% This is a sample manuscript marked up using the
%% AASTeX v6.31 LaTeX 2e macros.
%%
%% AASTeX is now based on Alexey Vikhlinin's emulateapj.cls
%% (Copyright 2000-2015).  See the classfile for details.

%% AASTeX requires revtex4-1.cls and other external packages such as
%% latexsym, graphicx, amssymb, longtable, and epsf.  Note that as of
%% Oct 2020, APS now uses revtex4.2e for its journals but remember that
%% AASTeX v6+ still uses v4.1. All of these external packages should
%% already be present in the modern TeX distributions but not always.
%% For example, revtex4.1 seems to be missing in the linux version of
%% TexLive 2020. One should be able to get all packages from www.ctan.org.
%% In particular, revtex v4.1 can be found at
%% https://www.ctan.org/pkg/revtex4-1.

%% The first piece of markup in an AASTeX v6.x document is the \documentclass
%% command. LaTeX will ignore any data that comes before this command. The
%% documentclass can take an optional argument to modify the output style.
%% The command below calls the preprint style which will produce a tightly
%% typeset, one-column, single-spaced document.  It is the default and thus
%% does not need to be explicitly stated.
%%
%% using aastex version 6.3
\documentclass[preprint2]{aastex631}

%% The default is a single spaced, 10 point font, single spaced article.
%% There are 5 other style options available via an optional argument. They
%% can be invoked like this:
%%
%% \documentclass[arguments]{aastex631}
%%
%% where the layout options are:
%%
%%  twocolumn   : two text columns, 10 point font, single spaced article.
%%                This is the most compact and represent the final published
%%                derived PDF copy of the accepted manuscript from the publisher
%%  manuscript  : one text column, 12 point font, double spaced article.
%%  preprint    : one text column, 12 point font, single spaced article.
%%  preprint2   : two text columns, 12 point font, single spaced article.
%%  modern      : a stylish, single text column, 12 point font, article with
%% 		  wider left and right margins. This uses the Daniel
%% 		  Foreman-Mackey and David Hogg design.
%%  RNAAS       : Supresses an abstract. Originally for RNAAS manuscripts
%%                but now that abstracts are required this is obsolete for
%%                AAS Journals. Authors might need it for other reasons. DO NOT
%%                use \begin{abstract} and \end{abstract} with this style.
%%
%% Note that you can submit to the AAS Journals in any of these 6 styles.
%%
%% There are other optional arguments one can invoke to allow other stylistic
%% actions. The available options are:
%%
%%   astrosymb    : Loads Astrosymb font and define \astrocommands.
%%   tighten      : Makes baselineskip slightly smaller, only works with
%%                  the twocolumn substyle.
%%   times        : uses times font instead of the default
%%   linenumbers  : turn on lineno package.
%%   trackchanges : required to see the revision mark up and print its output
%%   longauthor   : Do not use the more compressed footnote style (default) for
%%                  the author/collaboration/affiliations. Instead print all
%%                  affiliation information after each name. Creates a much
%%                  longer author list but may be desirable for short
%%                  author papers.
%% twocolappendix : make 2 column appendix.
%%   anonymous    : Do not show the authors, affiliations and acknowledgments
%%                  for dual anonymous review.
%%
%% these can be used in any combination, e.g.
%%
%% \documentclass[twocolumn,linenumbers,trackchanges]{aastex631}
%%
%% AASTeX v6.* now includes \hyperref support. While we have built in specific
%% defaults into the classfile you can manually override them with the
%% \hypersetup command. For example,
%%
%% \hypersetup{linkcolor=red,citecolor=green,filecolor=cyan,urlcolor=magenta}
%%
%% will change the color of the internal links to red, the links to the
%% bibliography to green, the file links to cyan, and the external links to
%% magenta. Additional information on \hyperref options can be found here:
%% https://www.tug.org/applications/hyperref/manual.html#x1-40003
%%
%% Note that in v6.3 "bookmarks" has been changed to "true" in hyperref
%% to improve the accessibility of the compiled pdf file.
%%
%% If you want to create your own macros, you can do so
%% using \newcommand. Your macros should appear before
%% the \begin{document} command.
%%

%% Reintroduced the \received and \accepted commands from AASTeX v5.2
%\received{March 1, 2021}
%\revised{April 1, 2021}
%\accepted{\today}

%% Command to document which AAS Journal the manuscript was submitted to.
%% Adds "Submitted to " the argument.
%\submitjournal{PSJ}

%% For manuscript that include authors in collaborations, AASTeX v6.31
%% builds on the \collaboration command to allow greater freedom to
%% keep the traditional author+affiliation information but only show
%% subsets. The \collaboration command now must appear AFTER the group
%% of authors in the collaboration and it takes TWO arguments. The last
%% is still the collaboration identifier. The text given in this
%% argument is what will be shown in the manuscript. The first argument
%% is the number of author above the \collaboration command to show with
%% the collaboration text. If there are authors that are not part of any
%% collaboration the \nocollaboration command is used. This command takes
%% one argument which is also the number of authors above to show. A
%% dashed line is shown to indicate no collaboration. This example manuscript
%% shows how these commands work to display specific set of authors
%% on the front page.
%%
%% For manuscript without any need to use \collaboration the
%% \AuthorCollaborationLimit command from v6.2 can still be used to
%% show a subset of authors.
%
%\AuthorCollaborationLimit=2
%
%% will only show Schwarz & Muench on the front page of the manuscript
%% (assuming the \collaboration and \nocollaboration commands are
%% commented out).
%%
%% Note that all of the author will be shown in the published article.
%% This feature is meant to be used prior to acceptance to make the
%% front end of a long author article more manageable. Please do not use
%% this functionality for manuscripts with less than 20 authors. Conversely,
%% please do use this when the number of authors exceeds 40.
%%
%% Use \allauthors at the manuscript end to show the full author list.
%% This command should only be used with \AuthorCollaborationLimit is used.

%% The following command can be used to set the latex table counters.  It
%% is needed in this document because it uses a mix of latex tabular and
%% AASTeX deluxetables.  In general it should not be needed.
%\setcounter{table}{1}

%%%%%%%%%%%%%%%%%%%%%%%%%%%%%%%%%%%%%%%%%%%%%%%%%%%%%%%%%%%%%%%%%%%%%%%%%%%%%%%%
%%
%% The following section outlines numerous optional output that
%% can be displayed in the front matter or as running meta-data.
%%
%% If you wish, you may supply running head information, although
%% this information may be modified by the editorial offices.
\shorttitle{What is the hottest cool star?}
\shortauthors{G\"unther et al.}

\graphicspath{{./}{figures/}}
%% This is the end of the preamble.  Indicate the beginning of the
%% manuscript itself with \begin{document}.

\begin{document}

\title{How hot can cool stars be?}

%% LaTeX will automatically break titles if they run longer than
%% one line. However, you may use \\ to force a line break if
%% you desire. In v6.31 you can include a footnote in the title.

%% A significant change from earlier AASTEX versions is in the structure for
%% calling author and affiliations. The change was necessary to implement
%% auto-indexing of affiliations which prior was a manual process that could
%% easily be tedious in large author manuscripts.
%%
%% The \author command is the same as before except it now takes an optional
%% argument which is the 16 digit ORCID. The syntax is:
%% \author[xxxx-xxxx-xxxx-xxxx]{Author Name}
%%
%% This will hyperlink the author name to the author's ORCID page. Note that
%% during compilation, LaTeX will do some limited checking of the format of
%% the ID to make sure it is valid. If the "orcid-ID.png" image file is
%% present or in the LaTeX pathway, the OrcID icon will appear next to
%% the authors name.
%%
%% Use \affiliation for affiliation information. The old \affil is now aliased
%% to \affiliation. AASTeX v6.31 will automatically index these in the header.
%% When a duplicate is found its index will be the same as its previous entry.
%%
%% Note that \altaffilmark and \altaffiltext have been removed and thus
%% can not be used to document secondary affiliations. If they are used latex
%% will issue a specific error message and quit. Please use multiple
%% \affiliation calls for to document more than one affiliation.
%%
%% The new \altaffiliation can be used to indicate some secondary information
%% such as fellowships. This command produces a non-numeric footnote that is
%% set away from the numeric \affiliation footnotes.  NOTE that if an
%% \altaffiliation command is used it must come BEFORE the \affiliation call,
%% right after the \author command, in order to place the footnotes in
%% the proper location.
%%
%% Use \email to set provide email addresses. Each \email will appear on its
%% own line so you can put multiple email address in one \email call. A new
%% \correspondingauthor command is available in V6.31 to identify the
%% corresponding author of the manuscript. It is the author's responsibility
%% to make sure this name is also in the author list.
%%
%% While authors can be grouped inside the same \author and \affiliation
%% commands it is better to have a single author for each. This allows for
%% one to exploit all the new benefits and should make book-keeping easier.
%%
%% If done correctly the peer review system will be able to
%% automatically put the author and affiliation information from the manuscript
%% and save the corresponding author the trouble of entering it by hand.

%\correspondingauthor{August Muench}
%\email{greg.schwarz@aas.org, gus.muench@aas.org}


\author[0000-0003-4243-2840]{Hans Moritz G{\"u}nther}
\affiliation{MIT Kavli Institute for Astrophysics and Space Research, 77 Massachusetts Avenue, Cambridge, MA 02139, USA}
\author{Co-Authors here}
\affiliation{author ordering will be decided later}
%% Note that the \and command from previous versions of AASTeX is now
%% depreciated in this version as it is no longer necessary. AASTeX
%% automatically takes care of all commas and "and"s between authors names.

%% AASTeX 6.31 has the new \collaboration and \nocollaboration commands to
%% provide the collaboration status of a group of authors. These commands
%% can be used either before or after the list of corresponding authors. The
%% argument for \collaboration is the collaboration identifier. Authors are
%% encouraged to surround collaboration identifiers with ()s. The
%% \nocollaboration command takes no argument and exists to indicate that
%% the nearby authors are not part of surrounding collaborations.

%% Mark off the abstract in the ``abstract'' environment.
\begin{abstract}
abstract here

\end{abstract}

%% Keywords should appear after the \end{abstract} command.
%% The AAS Journals now uses Unified Astronomy Thesaurus concepts:
%% https://astrothesaurus.org
%% You will be asked to selected these concepts during the submission process
%% but this old "keyword" functionality is maintained in case authors want
%% to include these concepts in their preprints.
%%\keywords{}

%% From the front matter, we move on to the body of the paper.
%% Sections are demarcated by \section and \subsection, respectively.
%% Observe the use of the LaTeX \label
%% command after the \subsection to give a symbolic KEY to the
%% subsection for cross-referencing in a \ref command.
%% You can use LaTeX's \ref and \label commands to keep track of
%% cross-references to sections, equations, tables, and figures.
%% That way, if you change the order of any elements, LaTeX will
%% automatically renumber them.
%%
%% We recommend that authors also use the natbib \citep
%% and \citet commands to identify citations.  The citations are
%% tied to the reference list via symbolic KEYs. The KEY corresponds
%% to the KEY in the \bibitem in the reference list below.

\section{Introduction} \label{sec:intro}
Stars across the main-sequence (MS) produce X-ray emission in two fundamentally different mechanisms. Cool stars have convective envelopes, thus they develop a solar-like dynamo which creates a magnetic field. In turn they show magnetic activity in a corona. Essentially all close-by late-type stars are X-ray emitters \citep{2004A&A...417..651S}. On the other end of the MS the most massive stars have fast winds. Instabilities in the winds heat the gas to a few MK. Again, nearly all of them are X-ray emitters \citep{1996A&AS..118..481B,1997A&A...322..167B}. Stars from mid-A to B operate neither mechanism: Their winds are too weak to produce detectable X-ray emission and their atmospheres are radiatively dominated and do not drive a solar-like dynamo. Mid-A to B-type stars thus are X-ray dark \citep{1997A&A...318..215S}.

It is no contradiction that some of these systems are seen in the ROSAT
All-Sky Survey (RASS) or in other X-ray datasets \citep{2020ApJ...902..114W}, because they often have unresolved late-type
companions. Due to the shorter lifetime of the A-type star the companion is
still at an early stage of its evolution and thus X-ray bright. The RASS
catalog contains 312 bright A-type stars. This is a detection rate of 10-15\% \citep{2007A&A...475..677S}. On closer examination most of sources turn out to be multiple \citep{2000A&A...359..227H,2003A&A...407.1067S}.

However, star spots with a low amplitude 0.05\% have been found on Vega \citep{2015A&A...577A..64B} together with a weak (disk-averaged line-of-sight component $< 1$~G) magnetic field \citep{2009A&A...500L..41L,2010A&A...523A..41P}.
Also, recent studies with \emph{Kepler} and \emph{TESS} do find rotational modulation of early A-type stars \citep{2011MNRAS.415.1691B,2017MNRAS.467.1830B,
2019MNRAS.487.4695S} and sometimes signatures of what seems to be magnetic flares \citep{2012MNRAS.423.3420B}. However, the latter can usually be attributed to binarity or artifacts such as contamination of the lightcurve by a near-by sources \citep{2017MNRAS.466.3060P}.

\cite{2002ApJ...579..800S} systematically observed mid-A type stars in the UV
looking for the subcoronal emission lines
of C\,{\sc iii} and O\,{\sc vi} formed between 50,000 and 300,000~K. They find a
very sharp cut-off, where stars with $T_\mathrm{eff}<8200$~K have line fluxes similar to our Sun (normalized to the bolometric
luminosity), but these lines are undetected in stars with $T_\mathrm{eff} > 8300$~K. It seems that that
transition between stars with and without a corona happens within 100~K and that the cut-ff temperature is compatible with theoretical predictions \citep{2000ASPC..210..187C,2002MNRAS.330L...6K}.

\citet{2002ApJ...579..800S} also analyzed archival \emph{ROSAT} observations
and find X-ray emission only for stars below 8200~K. However, the X-ray limits
are not very tight and in this paper we present new \emph{Chandra}/HRC-I observations that are about an order of magnitude more sensitive. \citet{2008ApJ...685..478N} extend the list of UV observations and do find  C\,{\sc iii} and O\,{\sc vi} in hotter sources, however, in those cases, the emission is better explained by late-type companions.
To avoid the problem of unresolved companions, we concentrate on well-studied A stars within 30~pc, where spectroscopy and recent planet searches with the radial-velocity method \citep[e.g.][]{2021AJ....161..157H}, direct imaging \citep[e.g.][]{2013ApJ...776....4N,2017AJ....154..245M}, and the study of annomalous GAIA proper motion \citep{2019A&A...623A..72K} essentially rule out the presence of a late-type stellar companion.

In section~\ref{sec:data}, we show the data from those new observations. We discuss the results in section~\ref{sec:discussion} and summarize our findings in section~\ref{sec:summary}.


\section{Data analysis} \label{sec:data}
We retrieved four Chandra/HRC-I observations from the Chandra archive. Details of the observations are listed in table~\ref{tab:obslog}. Data was reprocessed with CIAO 4.13 \citep{2006SPIE.6270E..1VF}, following standard analysis procedures.
For reproducibility, we provide the full analysis script\footnote{\url{https://github.com/hamogu/HottestCoolStar/blob/main/figures/HottestCoolStar.ipynb}}.

\begin{table*}
\caption{Chandra observations with pointing information \label{tab:obslog}}
\begin{tabular}{cccccc}
\hline \hline
target & obs. date & OBSID & RA (pointing) & Dec (pointing) & exp. time \\
 &  &  & $\mathrm{{}^{\circ}}$ & $\mathrm{{}^{\circ}}$ & $\mathrm{ks}$ \\
\hline
$\iota$ Cen & 2017-03-31 & 18930 & 200.1525 & -36.7119 & 9.7 \\
$\beta$ Leo & 2017-04-05 & 18931 & 177.2635 & 14.5719 & 10.1 \\
$\delta$ Leo & 2017-02-05 & 18932 & 168.5285 & 20.5240 & 10.1 \\
$\tau^3$ Eri & 2017-06-09 & 18933 & 45.5957 & -23.6223 & 19.9 \\
\hline
\end{tabular}
\end{table*}
First, we try to improve the astrometry of the Chandra observations.
The 90\% uncertainty circle for Chandra absolute astrometry is
0.8\arcsec\footnote{\url{https://cxc.cfa.harvard.edu/cal/ASPECT/celmon/}}. Since
the relative precision is even better than that, the astrometry can be
improved if a sufficient number of sources can be matched to a catalog
with high astrometric precision. We run the CIAO task \texttt{wavdetect} on the
X-ray data to detect X-ray sources with the intend to cross-match them
with 2MASS \citep{2006AJ....131.1163S} or GAIA
\citep{2016A&A...595A...1G,2018A&A...616A...1G}. However, there are
only few bright X-ray sources in the images; sources far from the
aimpoint are significantly broadened and thus measured X-ray
coordinates are less precise and the background is larger; only a
fraction of the X-ray sources is matched at all; and in some cases the
match is ambiguous. The brightest source in ObsID 18933 is \object{2MASS
  J03021318-2335198}, a Seyfert~1 galaxy, where the SIMBAD position of
the source is 0.7~arcsec to the East of the peak of the X-ray
emission, but a single cross-match is insufficent to fit a new coordinate system. Thus, we cannot improve the astrometry of the images beyond
the accuracy of the coordinates assigned by standard Chandra
processing.

\begin{figure*}
    \centering
    \includegraphics[width=\textwidth]{figures/chandra.pdf}
    \caption{Observations with the Chandra/HRC. The blue circles mark a 1~arcsec radius around the expected position of the target accounting for proper motion at the time of the observation. For $\tau^3$~Eri, the dotted circle accounts for the coordinate offset discussed in the text.
    \label{fig:chandra}}
\end{figure*}

Figure~\ref{fig:chandra} shows the Chandra/HRC-I images of our four
targets. Circle with 1~arcsec radius mark the position of the
target at the time of the observation. Coordinates and proper motions
are taken from \citet{2018yCat.1345....0G} for $\iota$~Cen and from
\cite{2007A&A...474..653V} for the remaining three targets. Only
$\tau^3$~Eri has significant emission within the marked circle in
Fig.~\ref{fig:chandra}. The expected position is about 0.7~arcsec to
the East of the peak of the X-ray emission; direction and distance of
the offset are very similar the offset detected in
2MASS~J03021318-2335198 and within the 90\% uncertainty expected for
the Chandra coordinates. We thus conclude that the apparent distance
is in fact due to the uncertainties of the Chandra coordinates and
that the source seen is a detection of $\tau^3$~Eri. For the remaining
three targets, no source is seen in Chandra at the expected position
or so close to it that it would be compatible with the coordinate
uncertainties.

Next, we determine source flux and uncertainties or upper limits.
We chose a source extraction region with 1.5~arcsec
radius to ensure that the source flux is captured, even in the
presence of coordinate uncertainties. The Chandra
point-spread-function (PSF) depends on the photon energy, but for
soft sources, as expected here, that region captures well above 95\%
of the PSF. Without knowing the source spectrum, we cannot fully
correct for the loss of photons outside the source aperture. For all targets, we calculate
uncertainties on the X-ray flux following a Bayesian approach that
takes into account the presence of a background following the method of \citet{1991ApJ...374..344K}. The background flux is determined from
a large region that is apparently source-free; in this way the
statistical error on the background rate is small.


However, there is a subtle difference between the upper end of a
confidence range and the upper limits for a detection. For those
sources that are undetected, we calculate the intensity that would be
required for an unambiguous detection. A source with 7 counts would be
detected with a significance of at least 99.7\% (corresponding to a
Gaussian-equivalent ``$3\sigma$'') with a probability of 0.5
\citep{2010ApJ...719..900K}.



Table~\ref{tab:detections} lists the detected count rate or upper limit. We convert the count rate into an energy flux, assuming a thermal spectrum. For Chandra/HRC-S,
1~ct~ks$^{-1}$ corresponds to an X-ray flux about $1.0*10^{-14}$~erg~s${-1}$ in the 0.1-5~keV band accroding to WebPIMMS\footnote{\url{https://heasarc.gsfc.nasa.gov/cgi-bin/Tools/w3pimms/w3pimms.pl}}; the variation around this vaue is no more than 15\% in the temperature range 0.8-13~MK. Using the distance and bolometric luminosity from \citet{2002ApJ...579..800S}, table~\ref{tab:detections} also lists $L_X$ and $\log(L_X/L_\mathrm{bol})$.
\begin{table*}
\caption{Limits and fluxes. Ranges given are 90\% confidence regions.\label{tab:detections}}
\begin{tabular}{ccccccc}
  \hline \hline
  & \multicolumn{4}{c}{90\% conf} & \multicolumn{2}{c}{99.7\% upper limit}\\
target & net. counts & net. rate & net. flux & $\log(L_X/L_\mathrm{bol})$ & lxlim & $\log(L_X/L_\mathrm{bol})$ \\
 & $\mathrm{ct}$ & $\mathrm{ct\,ks^{-1}}$ & $10^{-15}\mathrm{erg\,s^{-1}\,cm^{-2}}$ &  & $10^{26}\mathrm{erg\,s^{-1}}$ & $\mathrm{}$ \\
\hline
$\iota$ Cen & 0.0 ..  2.9 & 0.0 ..  0.3 & 0 ..   3 & $<-8.8$ & 2.2 & $<-8.5$ \\
$\beta$ Leo & 0.0 ..  3.0 & 0.0 ..  0.3 & 0 .. 2.9 & $<-9.1$ & 0.8 & $<-8.8$ \\
$\delta$ Leo & 0.0 ..  4.8 & 0.0 ..  0.5 & 0 .. 4.7 & $< -8.7$ & 2.1 & $<-8.6$ \\
$\tau^3$ Eri & 20.5 .. 39.8 & 1.0 ..  2.0 & 10 ..   20 & -7.6 & ... & ... \\
\hline
\end{tabular}
\end{table*}

All target stars are optically bright. We compare their $V$ magnitude
to Vega and scale the number of observed UV events from observations
of Vega \citep{2006ApJ...636..426P}. Based on this we expect 1 or fewer UV
events for each observation and we conclude that UV contamination is
negligible. For $\tau^3$ Eri, we checked the lightcurve and we do not find any significant variability, but --given the low-count number-- even a flare that doubles the X-ray ouput for a few ks could be hidden in the Poisson noise.

Our sample is designed to overlap with the UV sample of
\cite{2002ApJ...579..800S}, excluding any known binaries that cannot
be resolved in X-ray observations. \cite{2002ApJ...579..800S} present
upper limits from \emph{ROSAT} data for our for targets. Our new
limits (table~\ref{tab:detections}) for $\iota$~Cen, $\beta$~Leo, and
$\delta$~Leo are about an order of magnitude deeper than the
\emph{ROSAT} limits and $\tau^3$~Eri is now detected with a flux that
is compatible with the \emph{ROSAT} limit.

\section{Results and discussion}  \label{sec:discussion}
Table~\ref{tab} compares fluxes and upper limits that we determine here to similar A-type stars from the literature.


\begin{table*}
\caption{Stars of spectral type A with detailed X-ray observations \label{tab}}
\begin{tabular}{lcccccl}
  name      & age & age & SpT & $L_X$   &   log($L_X$/$L_{bol}$) & X-ray data\\
            & Myr & ref.&     & erg~s$^{-1}$  &  & ref. \\
\hline
HR 4796A    & 5-16 & 1,2,3   & A0  & $<1.3\times 10^{27}$  & $< -7.7$ & \citet{2014ApJ...786..136D}\\
Vega  & 100-500 & 4,5,6   & A0  & $<3.0\times 10^{25}$  & $< -10.0$& \citet{2006ApJ...636..426P}\\
$\iota$ Cen & 100-400 & 7,8 & A2Va & $<2\times 10^{26}$  & $< -8.5$ & this work\\
$\beta$ Leo & 30-70& 9, 10,11 & A3Va & $<8\times 10^{25}$  & $< -8.8$ & this work\\
$\delta$ Leo& 600-890 & 7,8 & A4IV &  $<2\times 10^{26}$  & $< -8.6$ & this work\\
$\tau^3$ Eri & 430-950 & 8 & A4V &  $8-16\times 10^{26}$  & -7.6 & this work\\
HR 8799   & 38-48 & 9    & A5  & $1.3\times10^{28}$   & -6.2     &  \citet{2010A\string&A...516A..38R}\\
$\beta$ Pic & 12-40 & 12,13 & A6  & $1.3\times10^{27}$   & -8.2  & \citet{2012ApJ...750...78G}\\
Altair    & 700-1000 & 7,14  & A7  & $1.4\times10^{27}$   & -7.4     &  {\citet{2009A\string&A...497..511R}}\\
Alderamin & 1000 & 15  & A7 & $1.4\times10^{27}$ & -7.5 & \cite{2002ApJ...579..800S}\\
% I don't think we need those in the table. They are Ap stars, which are different and can be discussed in the text. We don't list HAeBes either.
%\hline
%IQ Aur    & 63    & A0p & $4.0\times10^{29}$   & -6.5     &  {\citet{2011A\string&A...531A..58R}} \\
%$\alpha^2$ CVn & 200& A0p & $<1.0\times 10^{26}$  & $< -9.6$ &  {\citet{2011A\string&A...531A..58R}}\\
\hline
\end{tabular}\\
(1)~\citet{1999ApJ...512L..63W},
(2)~\citet{2013ApJ...767...96W},
(3)~\citet{2014ApJ...786..136D},
(4)~\citet{1998A&A...339..831B},
(5)~\citet{2010ApJ...712..250H},
(6)~\citet{2010ApJ...708...71Y},
(7)~\citet{2012AJ....143..135V},
(8)~\citet{2015ApJ...804..146D},
(9)~\citet{2015MNRAS.454..593B},
(10)~\citet{2019ApJ...870...27Z},
(11)~\citet{2019MNRAS.489.2189L},
(12)~\citet{2001ApJ...562L..87Z},
(13)~\citet{2010ApJ...723.1599M},
(14)~\citet{2018AJ....156..286S},
(15)~\citet{2009ApJ...701..209Z}
\end{table*}

Note that ages given in the table are often based on membership in moving groups. In some cases, membership is underdebate and stars could be considerable older if they were field stars, e.g.\ for $\beta$~Leo see the disscussion and references in \citet{2021AJ....161..186D}.


\subsection{X-ray spectrum}
It is of course difficult to constrain spectral properties of targets
are are undetected or only have very few photons. We do however know
something about the later A-type stars in table~\ref{tab}.  The X-ray
emission of \object{Altair}, spectral type A7, shows modest
variability on the 30\% level due to stellar rotation. The plasma
temperature around 1-4~MK \citep{2009A&A...497..511R}. Even earlier,
$\beta$~Pic has s spectral type of A6 and still weak detected X-ray
emission \citep{2005A&A...440..727H,2012ApJ...750...78G} best
described by coronal emission with a temperature around 1.1~MK. The
spectrum of \object{HR 8799} is equally soft with a temperature around
3~MK \citep{2010A\string&A...516A..38R}. We thus would expect any
X-ray emission from the other stars in table~\ref{tab} to have an
equally soft spectrum. Due to the limited energy resolution of the HRC
we cannot perform a spectral fit to the new data from
$\tau^3$~Eri. Fortunately, the energy flux derived from HRC count
rates or upper limits is insensitive to the plasam temperature for
cool plasma (section~\ref{sec:data}), so the fluxes and upper limits
given in table~\ref{tab} do not depend on the exact plasma
temperature.

\subsection{Debris disks}
Several of the stars in table~\ref{tab} host debris disks, which have
been confirmed in HR~4796A \citep[e.g.][]{1991ApJ...383L..79J}, Vega
\citep[e.g.][]{2005ApJ...628..487S}, $\iota$~Cen
\citep[e.g.][]{2011ApJ...736L..32Q}, $\beta$~Leo is a near-by star
with a bright dust disk \citep[e.g.][]{2021AJ....161..186D},
$\beta$~Pic \citep[e.g.][]{2001MNRAS.323..402L}.  These debris disks
form when planets or planetesimals in orbit around the star
collide. The smaller dust grains are blown out of the system through
radiation pressure relatively fast and need to be replenished
constantly by grinding down larger objects. While there is no
indication that the dust is responsible for the X-ray emission, it
highlights that spectral type and age might not be the only relevant
variables for the presence or absence of X-ray emission.

\subsection{Planets}
Since they are close and bright targets, the stars in table~\ref{tab}
have also been targeted by planet searches. Massive planets are
confirmed for HR~8799 \citep{2008Sci...322.1348M} and $\beta$~Pic
\citep[e.g.][]{2021AJ....161..179B}, but they are located at several
au distance from the star and are thus unlikely the influence the
X-ray emission from the star.

\subsection{Coronal and chromospheric activity}
\cite{2002ApJ...579..800S} present UV spectroscopy from a sample of
A-type stars that includes $\iota$~Cen, $\beta$~Leo, $\delta$~Leo, and
$\tau^3$~Eri in addition to other A-type stars not listed in our
table~\ref{tab}. They find that they can detect the UV emission lines
of C~{\sc iii} at 977\AA{} and 1175\AA{} as well as the O~{\sc vi}
doublet 1032/1037\AA{} in single stars up to spectral type A4 (which
they associate with $T_\mathrm{eff}=8200$~K). They also see those
lines in $\beta$~Ari, which has a primary component hotter than this
limit, but conclude that the most likely origin for the observed
emission is chromospheric activity in the cooler secondary in the
system. The emission lines of C~{\sc iii} and O~{\sc vi} are formed
below the corona; they have peak formation temperature between
50,000~K and 300,000~K. \cite{2002ApJ...579..800S} also analyse
archival \emph{ROSAT} data and find X-ray emission in all stars with
detectable C~{\sc iii} and O~{\sc vi} emission lines, except for
$\tau^3$~Eri. Conversely, they do not see X-ray emission in any of
their targets with $T_\mathrm{eff}>8300$~K, the regions where C~{\sc
  iii} and O~{\sc vi} are undetected. With our new, more sensitive
Chandra observations, we add the detection of $\tau^3$~Eri and place
tigther upper limtis on $\iota$~Cen, $\beta$~Leo, $\delta$~Leo, so
that we now see a one-to-one correspondance between X-ray emission and
C~{\sc iii} and O~{\sc vi} lines in the single stars from the
\cite{2002ApJ...579..800S} sample.

This confirms their conclusion of a very sharp drop in magnetic
activity at spectral type A4, where the $L_X$/$L_{bol}$ value drops by
an order of magnitude between $\tau^3$~Eri and
$\beta$~Leo. ($\delta$~Leo has luminosity class IV and might behave
differently, but if we disregard $\delta$~Leo for this argument, the
drop in X-ray flux is sharp.)



\subsection{Which process powers the X-ray emission from A stars?}
\subsubsection{Convective dynamos}
X-ray activity is common in later-type stars and for Altair
\citet{2009A&A...497..511R} concluded that the X-ray activity is
probably concentrated in the equatorial region, which is also
resposible for the generation of the chromosphere
\citep{1995ApJ...439.1011F}. Because Altair is a very fast rotator, it
has a considerable equatorial bulge and consequently lower effective
temperature, which might allow the formation of a thin convection zone
in that region. Similarly, \citet{2010A&A...516A..38R} argued that the
spectral classification of HR~8799 of A5 is based on metal lines,
while the atmospheric temperature might be better characterized by its
hydrogen lines, which would make it an F0 star
\citep{1999AJ....118.2993G}, cool enough to generate magnetic activity
through a convective dynamo. Similar to Altair, $\tau^3$~Eri is also a
rapid rotator with $v\sin i=180$~km~s${-1}$ and an esimated oblateness
(depending on the inclination angle $i$) around 7\%
\citep{2012A&ARv..20...51V}. On the other hand, $\delta$~Leo is an
equally fast rotator and remains undetected.

\subsubsection{Subsurface convection}
\citet{2019ApJ...883..106C} suggest the presence of sub-surface
convenction zones in A and late B stars. These thin zones are caused
by partial ionization of H and He; only a very small fraction of the
energy is transported by convection and thus they only cause very weak
surface magnetic fields. However, \citet{2019ApJ...883..106C} cannot
constrain the geometry of the field and thus it is unclear large-scale
fieds, which might be heating a transition region and a corona could
form in this scenario.

\subsubsection{Shear dynamos}
\citet{2014ApJ...786..136D} extensively discuss two types of shear
dynamos that could tranform some initial differential rotation of a
young star into a magnetic field that quickly decays, on time scales
of order of a Myr for a dynamo basedon on magnetic bouyancy
\citep{1995MNRAS.272..528T} or on times scales of order 300~yr for a
dynamo based on the Tayler instability
\citep{2002A&A...381..923S,2006A&A...449..451B}. Both scenarios
involve physical parameters that are uncertain by ordes of magnitude
and \citet{2014ApJ...786..136D} use their upper limit on the X-ray
flux from HR~4796A to exclude the most optimistic values for those
parameters. Our limit for $\beta$~Leo is significantly stricter, but
the star is also older (table~\ref{tab}), which places about the same
limits on the parameters for the \citet{1995MNRAS.272..528T} model as
does HR~4796A \citep[See Fig 3 in][]{2014ApJ...786..136D}.


\subsection{Special classes of A-type stars with X-ray emission}
\textbf{I think we need to discuss this somewhere, but I'm not happy with the location. Maybe shorten and move into introduction?}

In the following two subsections, we discuss two special classes of A-type stars where X-ray emission is commonly observed and how their differ from normal MS A-type stars.
\subsubsection{Chemically peculiar A stars}
%IQ Aur    & 63    & A0p & $4.0\times10^{29}$   & -6.5     &  {\citet{2011A\string&A...531A..58R}} \\
%$\alpha^2$ CVn & 200& A0p & $<1.0\times 10^{26}$  & $< -9.6$ &  {\citet{2011A\string&A...531A..58R}}\\


We mention Ap/Bp stars such as IQ~Aur as an exception to the rule of
X-ray dark B to mid-A stars.  They have strong magnetic fields and can
supposedly funnel their winds to collide in the equatorial plain,
forming shocks \citep{1997A&A...323..121B}.
\citet{2011A&A...531A..58R} observed two Ap stars of similar spectral
type (\object{IQ Aur} and \object{$\alpha^2$ CVn}), but with X-ray
fluxes differing by at least three orders of magnitude. IQ~Aur is a
hot coronal source with log($L_X$/$L_{bol}$=-6.5) that showed a
flare lasting a few hours, which can only be explained by magnetic
reconnection in addition to a funneled wind shock. It is unclear why
$\alpha^2$ CVn is so much fainter (log($L_X$/$L_{bol})<-9.6$),
despite a very similar mass, temperature, luminosity and magnetic
field strength. \citet{2011A&A...531A..58R} speculate that the wind
might be a temporary phenomenon that switches on or off for Ap stars
or that seemingly minor differences in the chemical composition
could cause this difference in X-ray properties.

\subsubsection{Pre Main-sequence A stars}
The predecessors of main-sequence (MS) stars in the mass range 2~M$_{\odot} < M_* < $~8~M$_{\odot}$ are called HAeBe stars (HAeBe). They are surrounded by a proto-stellar disk and can drive powerful jets.
X-ray surveys of HAeBes detect 70-100\% of the targets
\citep{2004ApJ...614..221S,2005ApJ...618..360H,2006A&A...457..223S,2009A&A...493.1109S,2020ApJ...888...15S}. This fraction is too high to be explained only with late-type companions and demonstrates that some HAeBes are intrinsic emitters, although they do not posses a convective layer and thus cannot drive a solar-like dynamo. X-ray luminosities range from a few $10^{29}$~erg/s to $10^{31}$~erg/s. The stars typically show $\log L_X/L_{bol}= -5$~to~$-7$ and X-ray emission drops sharply at the end of the HAeBe phase.

HAeBes have also been observed with grating spectroscopy. AB~Aur
\citep{2007A&A...468..541T} and HD~163296 \citep{2009A&A...494.1041G} both are early A-type stars and both show a luminosity of
$L_X=4\times 10^{29}$~erg/s. \emph{XMM-Newton} spectroscopy indicates
that the soft part of the spectrum of the HAeBe HD~163296 is not
formed at the stellar surface but a few stellar radii away, possibly
in a jet collimation shock \citep{2009A&A...494.1041G}.

An example for a later HAeBe (spectral type A8) is HD~104237 \citep{2004ApJ...614..221S,2008ApJ...687..579T}, which is considerably brighter and hotter than the other two.

We do not list HAeBe stars in table~\ref{tab} and we concentrate on targets that have evolved beyond the HAeBe stage in our discussion.


\section{Summary}
\label{sec:summary}
We present Chandra/HRC-I observations of four early A stars. $\tau^3$~Eri is clearly detected, while we set sensitive upper limits on the other three targets. With this detection and our new upper limits that are an order of magnitude better than previous \emph{ROSAT} data, there is now a one-to-one correspondance between X-ray emission and C\,{\sc iii} and O\,{\sc vi} lines formed between 50,000 and 300,000 K in the transition region \citep{2002ApJ...579..800S}. This confirms that both regions are powered by the same physical mechanism and that this mechanism essentially switches off between spectral types A3 and A4 as we observe a drop of X-ray luminosity by at least an order of magnitude. $\tau^3$~Eri is too old for fossil magnetic fields to play any role, so the magnetic field that powers X-ray emission and UV lines needs to be generated continuously. We disscuss thin sub-surface convection zones can be present in an otherwise radiative star and X-ray emission that is located mostly at the equatorial bulge.

\begin{acknowledgements}

This research has made use of data obtained from the Chandra Data Archive, and software provided by the Chandra X-ray Center (CXC) in the application package CIAO.
This research has made use of the SIMBAD database,
operated at CDS, Strasbourg, France \citep{2000A&AS..143....9W}. 
This research has made use of NASA’s Astrophysics Data System Bibliographic Services.
This publication makes use of data products from the Two Micron All Sky Survey, which is a joint project of the University of Massachusetts and the Infrared Processing and Analysis Center/California Institute of Technology, funded by the National Aeronautics and Space Administration and the National Science Foundation.
This work has made use of data from the European Space Agency (ESA) mission
{\it Gaia} (\url{https://www.cosmos.esa.int/gaia}), processed by the {\it Gaia}
Data Processing and Analysis Consortium (DPAC,
\url{https://www.cosmos.esa.int/web/gaia/dpac/consortium}). Funding for the DPAC
has been provided by national institutions, in particular the institutions
participating in the {\it Gaia} Multilateral Agreement.
HMG was supported by the National Aeronautics and Space Administration through Chandra Award Number GO9-20018X issued by the Chandra X-ray Observatory Center, which is operated by the Smithsonian Astrophysical Observatory for and on behalf of the National Aeronautics Space Administration under contract NAS8-03060.
\end{acknowledgements}

\facilities{Chandra/ACIS}

\software{AstroPy \citep{2013A&A...558A..33A,2018AJ....156..123A},
astroquery \citep{2019AJ....157...98G},
CIAO \citep{2006SPIE.6270E..1VF}, NumPy \citep{van2011numpy,harris2020array}, Matplotlib \citep{Hunter:2007}, Sherpa \citep{2007ASPC..376..543D,doug_burke_2021_4428938}}

\bibliography{bib}{}
\bibliographystyle{aasjournal}

%% This command is needed to show the entire author+affiliation list when
%% the collaboration and author truncation commands are used.  It has to
%% go at the end of the manuscript.
%\allauthors

%% Include this line if you are using the \added, \replaced, \deleted
%% commands to see a summary list of all changes at the end of the article.
%\listofchanges

\end{document}
